\chapter{附录一~~国际单位制(SI)}

我们知道,物理公式在确定物理量的数量关系的同时,也
确定了物理量的单位关系.因此,只要我们选定为数不多的
几个物理量的单位,就能够利用它们推导出其他物理量的单
位.这些被任意选定的物理量叫做\textbf{基本量},如力学中的长度、
质量和时间就是三个基本量,基本量的单位,如米、千克、秒
等,叫做基本单位,由基本量根据有关公式推导出来的其他
物理量,叫做\textbf{导出量},导出量的单位叫做\textbf{导出单位}.
\begin{table}[htp]
    \centering
    \caption{国际单位制的基本单位}
    \begin{tabularx}{\textwidth}{MMMM}
        \toprule[1.5pt]
        \multirow{2}*{物理量名称} & \multirow{2}*{单位名称} & \multicolumn{2}{c}{单位符号}       \\
                             &                     & 中文                       & 英文  \\
        \midrule
        长度                   & 米                   & 米                        & m   \\
        质量                   & 千克                  & 千克                       & kg  \\
        时间                   & 秒                   & 秒                        & s   \\
        电流                   & 安培                  & 安                        & A   \\
        热力学温度                & 开尔文                 & 开                        & K   \\
        发光强度                 & 坎德拉                 & 坎                        & cd  \\
        物质的量                 & 摩尔                  & 摩                        & mol \\
        \bottomrule[1.5pt]
    \end{tabularx}
\end{table}

所谓单位制,就是有关基本单位、导出单位等一系列单位的体制,由于所采用的基本量的不同,基本单位的不同,以及用来推导导出单位的定义公式的不同,存在着多种单位制.多种单位制并用,给科学技术的交流和发展带来不便,为了避免多种单位制的并存,国际上制订了一种通用的适合一切计量领域的单位制,叫做国际单位制,国际代号为SI.国际单位制是1960年第十一届国际计量大会通过的,其后并向全世界推荐使用,现在世界上许多国家采用了国际单位制或者正在向国际单位制过渡,我国也统一实行以国际单位制为基础的法定计量单位.

在力学范围内,国际单位制规定长度、质量和时间为三个基本量,它们的单位用米、千克、秒为基本单位,对于象热学、电磁学、光学等学科,除了上述三个基本单位外,还要加上另外的基本量,并选定合适的基本单位,才能导出其他物理量的单位.这样,国际单位制的基本单位共有七个.表1和表2分别列出了国际单位制的基本单位和常用的力学量的国际单位制单位.

\begin{table}\centering
    \belowrulesep=0pt
    \aboverulesep=0pt
    \caption{常用的力学量的国际单位制单位}
    \begin{tabularx}{\textwidth}{MM|MM|M}
        \Xhline{1.5pt}
        \multicolumn{2}{c}{物理量} & \multicolumn{2}{|c|}{单位} & \multirow{2}*{量纲式}                                                              \\
        名称                      & 符号                       & 名称                 & 国际符号                 &                                     \\
        \Xhline{1pt}
        面积                      & $S$                      & 平方米                & $\unit{m^{2}}$       & $[\mathrm{L}^2]$                    \\
        体积                      & $V$                      & 立方米                & $\unit{m^{3}}$       & $[\mathrm{L}^3]$                    \\
        位移                      & $s$                      & 米                  & $\unit{m}$           & $[\mathrm{L}]$                      \\
        速度                      & $v$                      & 米每秒                & $\unit{m/s}$         & $[\mathrm{L}\mathrm{T}^{-1}]$       \\
        加速度                     & $a$                      & 米每二次方秒             & $\unit{m/s^{2}}$     & $[\mathrm{LT}^{-2}]$                \\
        角速度                     & $\omega$                 & 弧度每秒               & $\unit{rad/s}$       & $[\mathrm{T}^{-1}]$                 \\
        角加速度                    & $d$                      & 弧度每二次方秒            & $\unit{rad/s^{2}}$   & $[\mathrm{T}^{-2}]$                 \\
        转速                      & $n$                      & 1每秒                & $\unit{s^{-1}}$      & $[\mathrm{T}^{-1}]$                 \\
        频率                      & $\nu,\; f$               & 赫兹                 & \unit{Hz}            & $[\mathrm{T}^{-1}]$                 \\
        密度                      & $\rho$                   & 千克每立方米             & $\unit{kg/m^{3}}$    & $[\mathrm{L}^{-3}\mathrm{M}]$       \\
        力                       & $F$                      & 牛顿                 & \unit{N}             & $[\mathrm{LMT}^{-2}]$               \\
        重量                      & $G$                      & 牛顿                 & \unit{N}             & $[\mathrm{LMT}^{-2}]$               \\
        力矩                      & $M$                      & 牛顿米                & $\unit{N\cdot m}$    & $[\mathrm{L^2MT}^{-2}]$             \\
        动量                      & $p$                      & 千克米每秒              & $\unit{kg\cdot m/s}$ & $[\mathrm{LMT}^{-1}]$               \\
        冲量                      & $I$                      & 牛顿秒                & $\unit{N\cdot s}$    & $[\mathrm{LMT}^{-1}]$               \\
        压强                      & $p$                      & 帕斯卡                & \unit{Pa}            & $[\mathrm{L}^{-1}\mathrm{MT}^{-2}]$ \\
        功                       & $W$                      & 焦耳                 & \unit{J}             & $[\mathrm{L}^{2}\mathrm{MT}^{-2}]$  \\
        能                       & $E$                      & 焦耳                 & \unit{J}             & $[\mathrm{L}^{2}\mathrm{MT}^{-2}]$  \\
        功率                      & $P$                      & 瓦特                 & \unit{W}             & $[\mathrm{L}^{2}\mathrm{MT}^{-3}]$  \\
        \Xhline{1.5pt}
    \end{tabularx}
\end{table}





\chapter{附录二~~量纲}

我们知道,物理量可分为基本量和导出量,既然导出量可以从基本量导出,那么每个导出量一定可以用基本量的某种组合表示出来.表示一个物理量是由哪些基本量组成和怎样组成的式子,叫做这个物理量的量纲式,在国际单位制中,所有的力学量都是由长度、质量、时间这三个基本量组成的,如果用$L,M,T$分别表示这三个基本量,那么,一个物理量$Q$的量纲式的一般形式就是
\[[Q]=\si{[L^{\alpha}M^{\beta}T^{\gamma}]}\]
其中$\alpha, \beta, \gamma$分别叫做物理量$Q$对于长度、质量、时间的\textbf{量纲}.

下面举出几个物理量的量纲式:
\begin{itemize}
    \item 体积的量纲式
          \[[V]=\si{[L^3]}  \]
    \item 速度的量纲式
          \[[v]=\frac{[s]}{[t]}=\si{[LT^{-1}]}  \]
    \item 加速度的量纲式
          \[ [a]=\frac{[v]}{[t]}=\si{[LT^{-2}]} \]
    \item 力的量纲式
          \[ [F]=[m][a]=\si{[LMT^{-2}]} \]
\end{itemize}

在上面的表2中列出了一些力学量的量纲式.

物理量的量纲式可以用来检验物理关系式的正确性.检验时所根据的原则是:只有量纲式相同的项才能相加减,而且等号两边一定要有相同的量纲式.一个正确的物理关系式,一定是符合上述原则的.



















