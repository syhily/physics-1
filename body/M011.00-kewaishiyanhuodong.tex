\chapter{课外实验活动}
\setcounter{section}{16}
\section{测量尼龙丝的抗断拉力}
在商店里买来了尼龙丝,但不知道它的抗断拉力.如果你手边只有一个质量为1千克的重物和一只量角器,你能测出尼龙丝的抗断拉力吗?

如图10.22所示,把1千克的重物挂在一段尼龙丝的中点.根据力的平行四边形法则,你可以找出$F$和$G$的关系式.

用手拉着尼龙丝的另一端,并沿着箭头$a$所示的方向拉尼龙丝,当逐渐拉紧尼龙丝时,$\alpha$角增大,力$F$也随着增大.用量角器量自尼龙丝刚刚被扯断时的$\alpha$角,就可以知道尼龙丝的抗断拉力.

\begin{figure}[htp]\centering
    \begin{tikzpicture}[>=latex,scale=1]
        \fill [pattern =north east lines](-2, 1) rectangle (-2-.25, 1-.5);
        \draw [thick](-2,1)--(-2,.5);
        \draw (0,0)--(-2,.75);
        \draw (0,0)--(2,1);
        \draw [thick,->](0,0)--(-1,.75/2)node [below]{$F$};
        \draw [thick,->](0,0)--(1,1/2)node [below]{$F$};
        \draw [thick,->] (1.75,1.4)--node [above]{$a$} (2.5,1.4);
        \draw (0,0)--(0,-1);
        \draw (-0.25,-1) rectangle(.25, -1.8);
        \draw [thick,->](0,-2.8/2)--(0,-2.5) node [right]{$G=mg$};
        \draw (1/2,1/4) arc  (30:160:.56);
        \node at (0,.25){$\alpha$};
    \end{tikzpicture}
    \caption{}
\end{figure}
\newpage
\section{滴水法测重力加速度}

利用水滴下落可以测出重力加速度,调节水龙头,让水一滴一滴地流出,在水龙头正下方放一个盘子,使水滴落到盘子上,要把盘子垫起来,以便能清晰地听到水滴碰到盘子的响声.

细心地调整阀门,使第一个水滴碰到盘子的瞬间,第二个水滴正好从阀门处开始下落.你一边听水滴碰盘子的响声,一边注视着阀门处的水滴,就很容易做到这一点.这样调整好之后,水滴从阀门落到盘子经过的时间,就正好等于相继滴下的两个水滴之间的时间间隔.

数出在半分钟或一分钟内滴下的水滴的数目,或者测出下落50—100个水滴经过的时间,就可以算出水滴下落的时间$t$.用米尺量出水滴下落的距离$h$.将$t$、$h$值代入公式$h=\dfrac{1}{2}gt^2$中,就可以计算出重力加速度$g$.

\section{用秒表测量玩具手枪子弹射出的速度}
根据你学过的竖直上抛运动的知识,用一只秒表就可以简便地测出玩具手枪子弹射出的速度.

让子弹从枪口竖直向上射出,用秒表测出子弹从射出枪口到落回原地经过的时间$t$.设子弹射出的速度为$v_0$,子弹从射出到落回原地所用的时间
\[t=\frac{2v_0}{g}\]
由此可以求出子弹射出的速度
\[v_0=\frac{gt}{2}\]

用这种方法测出玩具手枪子弹射出的速度.

\section{用尺测量玩具手枪子弹射出的速度}
根据你学过的平抛运动的知识,用尺可以简便地测出玩具手枪子弹射出的速度.

让子弹从高度为$h$的地方水平射出,用卷尺量出子弹落地处到射自处的水平距离$l$和高度$h$.如果子弹的射出速度$v_0$,那么,
\[\begin{split}
        h & =\frac{1}{2}gt^2 \\
        l & =v_0t
    \end{split}\]
由此可以求出子弹射出的速度
\[v_0=l \sqrt{\frac{g}{2h}}\]

用这种方法测出玩具手枪子弹射出的速度.

\section{估测自行车受到的阻力}

骑自行车时,如果停止用力蹬脚踏板,由于受到阻力,自行车在水平路面上前进一段路程就停下来.设计一个实验,测量自行车在这段路程里所受的平均阻力.在这个实验里,你要测量些什么?实际测量一下,你自己或你的同学骑自行车停下来时,在这段路程里受到的平均阻力是多少?
\newpage
\section{验证向心力公式}
用下面的方法可以验证向心力公式.如图10.23那样,把尼龙绳穿过圆珠笔杆,在绳的两端分别拴上大小不同的两个石块.手握笔杆,抡动小石块,使它做匀速圆周运动,并且使大石块的位置保持基本上不动.这时使小石块做匀速圆周运动的向心力就等于大石块的重量(想一想,为什么).把小石块转动的半径$r$改变三次,测出每次的$r$和每次小石块转动20圈所用的时间$t$.算出小石块各次转动的角速度$\omega$,再测出大石块的质量$M$和小石块的质量$m$.利用以上测得的数据算出每次小石块做匀速圆周运动所需的向心力$mr\omega^2$,看看是否都等于大石块的重量$Mg$.
\begin{figure}[htp]\centering
    \begin{tikzpicture}[>=latex,scale=1]
        \newcommand\dynanometer[3][0]{
            \begin{scope}[#2,rotate=#1,inner sep=0pt]
                \coordinate (A) at (0,2.0);
                \coordinate (B) at (0,2.0+#3*0.2);
                \foreach \x/\y in {100/2.0, 70/1.5, 50/1.2, 30/0.7 }
                    {
                        \draw[line width=\y pt,gray!\x ]([yshift=1cm]B)circle(0.2);
                    }
                \fill[top color=gray, bottom color=gray,middle color=white]([xshift=-1mm,yshift=7mm]B)rectangle([xshift=1mm,yshift=8.7mm]B);
                \fill[green!30!black]([xshift=2.8mm,yshift=6mm]B)--([xshift=2.8mm,yshift=-1.2cm]B)to[bend left=50]([xshift=-2.8mm,yshift=-1.2cm]B)--([xshift=-2.8mm,yshift=6mm]B)to[bend left=50]cycle;
                \fill[lightgray,even odd rule](0.1,0.4)arc(360:180:0.1)--++(0,1.6)--++(0.2,0)--cycle(0,0.4)circle(0.05)([xshift=-0.4mm]A)rectangle++(0.8mm,0.5mm);
                \fill[lightgray!30,even odd rule]([xshift=2.2mm,yshift=5mm]B)--([xshift=2.2mm,yshift=-1.1cm]B)to[bend left]([xshift=-2.2mm,yshift=-1.1cm]B)[rounded corners=2pt]--([xshift=-2.2mm,yshift=5mm]B)[sharp corners]--([xshift=-1mm,yshift=5mm]B)arc(180:0:0.1)[rounded corners=2pt]--cycle[sharp corners]
                ([xshift=-0.4mm,yshift=2mm]B)rectangle([xshift=0.4mm]A);
                \draw[darkgray](0.1,0.1)arc(360:90:0.1)--++(0,0.15);
                \foreach \x in {0,1,...,9}
                    {
                        \draw[line width=0.1pt,line join=round,lightgray]([yshift=2mm-\x*0.15mm-\x*#3*0.2mm]B)--++(0.3mm,-0.0375mm-#3*0.05mm)--++(-0.6mm,-0.075mm-#3*0.1mm)--++(0.3mm,-0.0375mm-#3*0.05mm);
                    }
                \fill[red]([yshift=-0.2mm]A)--([xshift=-1mm]A)--([xshift=1mm]A);
                \foreach \x in {0,1,2,3,4}
                    {
                        \draw[ultra thin]([xshift=0.5mm,yshift=-\x*2mm]B)--++(0.07,0)node[rotate=#1,right]{\resizebox{!}{0.5mm}{\x}};
                        \draw[ultra thin]([xshift=-0.5mm,yshift=-\x*2mm]B)--++(-0.07,0)node[rotate=#1,left]{\resizebox{!}{0.5mm}{\x}};
                        \foreach \y in {1,2,3,4,6,7,8,9}
                            {
                                \draw[ultra thin]([xshift=0.5mm,yshift=-\x*2mm-\y*0.2mm]B)--++(0.05,0);
                                \draw[ultra thin]([xshift=-0.5mm,yshift=-\x*2mm-\y*0.2mm]B)--++(-0.05,0);
                            }
                        \draw[ultra thin]([xshift=0.5mm,yshift=-\x*2mm-1mm]B)--++(0.06,0);
                        \draw[ultra thin]([xshift=-0.5mm,yshift=-\x*2mm-1mm]B)--++(-0.06,0);
                    }
                \draw[ultra thin]([xshift=0.5mm,yshift=-1cm]B)--++(0.07,0)node[rotate=#1,right]{\resizebox{!}{0.5mm}{5}};
                \draw[ultra thin]([xshift=-0.5mm,yshift=-1cm]B)--++(-0.07,0)node[rotate=#1,left]{\resizebox{!}{0.5mm}{5}};
            \end{scope}
        }
        % \useasboundingbox(-1.9,-3.55)rectangle(3,3.15);
        % \dynanometer{scale=0.7}{1}
        \draw(0,1)--(0,3);
        \fill[ball color=green!50!black](0,1)circle(0.2);
        % \draw[thick](-0.5,-0.5)--(0.5,-0.5);
        % \fill[pattern=north east lines]  (-0.5,-0.5)rectangle(0.5,-0.75);

        \fill[pink!10!orange!10,draw=black,very thin]
        ( 0.802,3.496)..controls( 0.660,3.493)and( 0.580,3.513)..
        ( 0.548,3.583)..controls( 0.520,3.628)and( 0.448,3.741)..
        ( 0.401,3.773)..controls( 0.286,3.821)and( 0.213,3.837)..
        ( 0.134,3.819)..controls( 0.068,3.797)and( 0.014,3.822)..
        (-0.074,3.810)..controls(-0.187,3.797)and(-0.213,3.787)..
        (-0.219,3.711)..controls(-0.222,3.666)and(-0.203,3.641)..
        (-0.163,3.613)..controls(-0.156,3.487)and(-0.081,3.361)..
        (-0.012,3.232)..controls( 0.067,3.206)and( 0.130,3.182)..
        ( 0.183,3.151)..controls( 0.291,3.085)and( 0.340,3.101)..
        ( 0.421,3.161)..controls( 0.411,3.136)and( 0.424,3.121)..
        ( 0.466,3.113)..controls( 0.526,3.110)and( 0.576,3.088)..( 0.621,3.092);
        \fill[left color=blue!60!black,right color=blue!60!black,middle color=white](-0.03,3)rectangle(0.03,4.1);
        \fill[left color=blue!60!black,right color=blue!60!black,middle color=white](0.01,4.2)--(0.03,4.1)--(-0.03,4.1)--(-0.01,4.2);
        \fill[pink!10!orange!10,draw=black,very thin]
        ( 0.177,3.737)..controls( 0.126,3.826)and( 0.066,3.845)..
        (-0.033,3.916)..controls(-0.130,3.970)and(-0.107,4.095)..
        (-0.034,4.051)..controls( 0.025,4.009)and( 0.045,4.029)..
        ( 0.083,4.011)..controls( 0.102,4.000)and( 0.128,3.958)..
        ( 0.154,3.942)..controls( 0.246,3.898)and( 0.318,3.826)..( 0.401,3.773)
        (-0.085,3.756)..controls(-0.049,3.743)and( 0.011,3.694)..
        ( 0.035,3.703)..controls( 0.059,3.703)and( 0.071,3.702)..
        ( 0.111,3.671)..controls( 0.154,3.630)and( 0.160,3.566)..
        ( 0.117,3.560)..controls( 0.077,3.571)and( 0.040,3.574)..
        (-0.007,3.572)..controls(-0.075,3.575)and(-0.107,3.561)..(-0.163,3.613)
        (-0.163,3.613)..controls(-0.182,3.589)and(-0.214,3.532)..
        (-0.194,3.498)..controls(-0.185,3.478)and(-0.183,3.461)..
        (-0.177,3.447)..controls(-0.193,3.385)and(-0.174,3.333)..
        (-0.129,3.321)..controls(-0.140,3.300)and(-0.142,3.274)..
        (-0.121,3.251)..controls(-0.105,3.231)and(-0.089,3.204)..
        (-0.040,3.204)..controls(-0.007,3.205)and(-0.008,3.220)..
        ( 0.025,3.227)..controls( 0.090,3.244)and( 0.107,3.276)..
        ( 0.144,3.291)..controls( 0.174,3.306)and( 0.178,3.311)..
        ( 0.176,3.342)..controls( 0.174,3.366)and( 0.163,3.381)..
        ( 0.140,3.383)..controls( 0.177,3.393)and( 0.190,3.421)..
        ( 0.184,3.460)..controls( 0.180,3.470)and( 0.178,3.472)..
        ( 0.152,3.468)..controls( 0.175,3.476)and( 0.185,3.490)..
        ( 0.184,3.508)..controls( 0.182,3.528)and( 0.169,3.549)..
        ( 0.117,3.560)..controls( 0.077,3.571)and( 0.040,3.574)..
        (-0.007,3.572)..controls(-0.075,3.575)and(-0.107,3.561)..(-0.163,3.613);
        \draw[very thin]
        ( 0.072,3.581)..controls( 0.060,3.646)and( 0.079,3.660)..( 0.122,3.641)
        ( 0.129,3.534)..controls( 0.078,3.534)and( 0.094,3.491)..( 0.139,3.470)
        ( 0.100,3.365)..controls( 0.088,3.332)and( 0.094,3.311)..( 0.117,3.296)
        (-0.177,3.447)..controls(-0.054,3.438)and( 0.053,3.448)..( 0.151,3.468)
        (-0.129,3.321)..controls(-0.112,3.315)and(-0.037,3.331)..
        ( 0.027,3.349)..controls( 0.085,3.369)and( 0.115,3.370)..( 0.140,3.383)
        ( 0.127,3.454)..controls( 0.109,3.425)and( 0.117,3.406)..( 0.149,3.395)
        ( 0.189,3.677)..controls( 0.195,3.606)and( 0.211,3.550)..( 0.232,3.522)
        ( 0.262,3.475)..controls( 0.294,3.394)and( 0.399,3.410)..( 0.426,3.358)
        ( 0.491,3.264)..controls( 0.488,3.240)and( 0.465,3.235)..( 0.446,3.216)
        ( 0.589,3.245)..controls( 0.620,3.234)and( 0.654,3.229)..( 0.676,3.231)
        ( 0.599,3.276)..controls( 0.655,3.258)and( 0.685,3.255)..( 0.713,3.257)
        ( 0.538,3.358)..controls( 0.542,3.371)and( 0.555,3.382)..( 0.568,3.384)
        (-0.061,4.052)..controls(-0.091,4.028)and(-0.029,4.015)..(-0.023,3.988)
        ( 0.016,4.015)..controls( 0.005,4.004)and( 0.007,3.989)..( 0.008,3.979)
        ( 0.045,3.980)..controls( 0.057,3.987)and( 0.061,3.997)..( 0.061,4.011)
        ( 0.069,3.970)..controls( 0.081,3.979)and( 0.084,3.990)..( 0.084,4.001);
        \draw[densely dashed](0,4.2)ellipse(2 and 0.5);
        \draw(0,4.2)--(2,4.2);
        \fill[ball color=red](2,4.2)circle(0.1);

    \end{tikzpicture}
    \caption{}
\end{figure}

要注意,一定要把石块拴牢靠,以免实验时石块飞出,发生意外.

\section{制作杆秤}
学习了物体平衡的知识,你可以自己制作一把杆秤.

取一根30—50厘米长的细木棍作秤杆,一个质量1千克左右的物体作秤锤.照图6.26那样先确定秤钩和提纽的位置.然后在秤钩不挂物体的情况下,把秤锤挂在秤杆上,提起提纽,使秤杆平街,这时秤锤的位置就是秤的零刻度$A$点(这点也叫定盘星).再把质量为1千克的物体挂在秤钩上,调整秤锤的位置,使秤杆平衡,这时秤锤的位置就是秤的1千克刻度点.再在秤钩上挂质量为2千克、3千克的物体,使秤杆平衡,找出2千克、3千克刻度的位置.你将发现这几个刻度间的距离是均匀的(为什么,请同学们自己证明).根据这个规律,你可以在秤杆上找出4千克、5千克等刻度的位置,把每千克刻度间的距离等分成10份,每份问的距离就代表0.1千克,这样你的杆秤就做成了.

把你制作的杆秤跟商店里层的秤核对一下,看看你的杆秤用起来准不准?

如果要增大杆秤的称量范围,想一想应该怎么办?
\newpage
\section{研究小球滚下的位置}
如图10.24所示,让小球从斜面上某一位置滚下,如果小球在运动中受到的摩擦阻力很小,可以忽略不计,你能否预计出小球落在地面上的位置?在你预计的位置上放一个塑料杯子,看看小球滚下时是否落入杯中,这个实验说明了什么问题?
\begin{figure}[htp]\centering
    \begin{tikzpicture}[>=stealth,scale=1]
        \fill [pattern=north east lines](-4.5,-.25) rectangle (3,0);
        \draw [thick](-4.5,0)--(3,0);
        \draw [thick,fill=lightgray](-1.5,2) --(1.7,2.92)--(1.7,2);
        \draw [semithick,fill=brown!70](-2.2,1.8)rectangle(2.2,2);
        \draw [semithick,fill=brown](-1.9,1.8)rectangle(1.9,1.6);
        \draw [semithick,fill=brown](-2.0,1.8)--(-1.95,0)--(-1.85,0)--(-1.8,1.8)--cycle;
        \draw [semithick,fill=brown](2.0,1.8)--(1.95,0)--(1.85,0)--(1.8,1.8)--cycle;
        \fill [ball color=gray,semithick](1.5,3.0) circle (.15);
        \fill [left color=gray,right color=gray,middle color=white](-4.2,0.4)--(-3.8,0.4)--(-3.9,0)--(-4.1,0)--cycle;
        \fill [darkgray] (-4,0.4) ellipse(0.2 and 0.05);
        \fill [lightgray] (-4.2,0.4) arc( 180:-50:0.2 and 0.05)to[bend right=20]cycle;
    \end{tikzpicture}
    \caption{}
\end{figure}



















